\documentclass[10pt,a4paper, margin=1in]{article}
\usepackage{fullpage}
\usepackage{amsfonts, amsmath, pifont}
\usepackage{amsthm}
\usepackage{graphicx}
\usepackage{tkz-euclide}
\usepackage{tikz}
\usepackage{pgfplots}

\usepackage{geometry}
 \geometry{
 a4paper,
 total={210mm,297mm},
 left=10mm,
 right=10mm,
 top=10mm,
 bottom=10mm,
 }
 % Write both of your names here. Fill exxxxxxx with your ceng mail address.
 \author{
  Karaman, Arda\\
  \texttt{e2237568@ceng.metu.edu.tr}
  \and
  Fulser, Göktürk\\
  \texttt{e2237386@ceng.metu.edu.tr}
}
\title{CENG 384 - Signals and Systems for Computer Engineers \\
Spring 2021 \\
Homework 4}
\begin{document}
\maketitle



\noindent\rule{19cm}{1.2pt}

\begin{enumerate}

\item %write the solution of q1
    \begin{enumerate}
    % Write your solutions in the following items.
    \item %write the solution of q1a
    \begin{align*}
        g(t) = 4x(t) - 5y(t) + \int(x(t)-6y(t))dt\\
        \text{and } y(t) \int g(t)dt = \int[4x(t) - 5y(t) + \int(x(t)-6y(t)dt]dt\\
    \end{align*}
Differentiate it once
    \begin{align*}
        \frac{dy(t)}{dt} = 4x(t) - 5y(t) + \int(x(t)-6y(t))dt\\
    \end{align*}
Differentiate it once more
    \begin{align*}
        \frac{d^2y(t)}{dt^2} = \frac{4dx(t)}{dt} - \frac{5dy(t)}{dt} + x(t)-6y(t)\\
    \end{align*}
So differential equation is;
    \begin{align*}
        \frac{d^2y(t)}{dt^2} + \frac{5dy(t)}{dt} +6y(t) = \frac{4dx(t)}{dt} + x(t)
    \end{align*}
    
    
    
    
    \item %write the solution of q1b
    \begin{align*}
        (j\omega)^2 Y(j\omega) + 5 (j\omega) Y(j\omega) + 6Y(j\omega) = 4(j\omega)X(j\omega) + X(j\omega)\\
    \end{align*}
So Frequency Response is;
    \begin{align*}
        Y(j\omega)[ 6 + 5j\omega - \omega^2 ] = X(j\omega)[ 1+ 4j\omega ]
    \end{align*}
    
    
    
    
    
    \item %write the solution of q1c
    \begin{align*}
        \frac{Y(j\omega)}{X(j\omega)} = \frac{1+4j\omega}{6-\omega^2+5j\omega}\\
    \end{align*}
By using Partial Fraction;
    \begin{align*}
        H(j\omega) = \frac{Y(j\omega)}{X(j\omega)} = \frac{7}{2 + j\omega} - \frac{3}{3 + j\omega}\\
    \end{align*}
By applying inverse Fourier transform;
    \begin{align*}
        h(t) = (7e^{-2t} - 3e^{-3t})u(t)
    \end{align*}
    \item %write the solution of q1d
    \begin{align*}
        x(t) = \frac{1}{4} e^{\frac{-t}{4}}u(t)\\
    \end{align*}
By applying Fourier Transform;
    \begin{align*}
        X(j\omega) = \frac{1}{4}(\frac{1}{\frac{1}{4}+j\omega})\\
        \frac{Y(j\omega)}{X(j\omega)} = (\frac{7}{2+j\omega} - \frac{3}{3+j\omega})\\
        Y(j\omega) = (\frac{7}{2+j\omega} - \frac{3}{3+j\omega})\cdot \frac{1}{4}(\frac{1}{\frac{1}{4}+j\omega})\\
        Y(j\omega) = \frac{1}{4}[(\frac{7}{2+j\omega})\cdot (\frac{1}{\frac{1}{4}+j\omega}) - (\frac{3}{3+j\omega})\cdot (\frac{1}{\frac{1}{4}+j\omega})]\\
        Y(j\omega) = \frac{1}{4} [\frac{4}{\frac{1}{4}+j\omega} - \frac{4}{2+j\omega} - \frac{1.090}{\frac{1}{4}+j\omega} + \frac{1.090}{3 + j\omega}]\\
        Y(j\omega) = \frac{1}{4}[\frac{2.9}{\frac{1}{4}+j\omega} - \frac{4}{2+j\omega} + \frac{1.09}{3 + j\omega}]\\
    \end{align*}
By applying inverse Fourier Transform;
    \begin{align*}
        y(t) = \frac{1}{4}(2.9e^{\frac{-t}{4}} - 4e^{-2t} + 1.09e^{-3t})u(t)\\
        y(t) = (0.72e^{\frac{-t}{4}} - e^{-2t} + 0.27e^{-3t})u(t)
    \end{align*}
    \end{enumerate}




\item %write the solution of q2
    \begin{enumerate}
    % Write your solutions in the following items.
    \item %write the solution of q2a
    \begin{align*}
        \frac{Y(j\omega)}{X(j\omega)} = \frac{j\omega + 4}{ (j\omega)^2 + 5j\omega + 6} \\
        Y(j\omega) \cdot (j\omega)^2 + Y(j\omega) \cdot 5j\omega + Y(j\omega) \cdot 6 = X(j\omega) \cdot j\omega + X(j\omega) \cdot 4 
    \end{align*} \\
    
    By applying inverse Fourier transform;
    \begin{align*}
        \frac{d^2(y)}{dt^2} + 5 \cdot \frac{d(y)}{dt} + 6 \cdot y(t) = \frac{d(x)}{dt} + x(t) \cdot 4
    \end{align*} \\
    
    So the answer becomes;
    \begin{align*}
        y \prime \prime + 5 \cdot y\prime + 6 \cdot y = x\prime + 4\cdot x
    \end{align*}
    
    
    
    \item %write the solution of q2b
    \begin{align*}
        H(j\omega) = \frac{j\omega + 4}{ (j\omega)^2 + 5j\omega + 6} \\
        H(j\omega) = \frac{j\omega + 4}{(j\omega + 2) \cdot (j\omega + 3)} \\
        H(j\omega) = \frac{2}{(j\omega + 2)} - \frac{1}{(j\omega + 3)}
    \end{align*}
    
    By applying inverse Fourier transform;
    \begin{align*}
        h(t) = 2 e^{-2t} \mu(t) + e^{-3t} \mu(t) \\
        \text{So the answer becomes; } h(t) = (2 e^{-2t} + e^{-3t}) \cdot \mu(t) 
    \end{align*}
    
    
    
    \item %write the solution of q2c
    \begin{align*}
        Y(j\omega) = X(j\omega) \cdot H(j\omega) \\
        X(j\omega) = e^{-4t} \mu(t) - t \cdot e^{-4t} \mu(t) \text{ is given.}
    \end{align*}
    
    By applying inverse Fourier transform;
    \begin{align*}
        X(j\omega) = \frac{1}{(j\omega + 4)} - \frac{1}{(j\omega + 4)^2} \\
        X(j\omega) = \frac{j\omega + 4 - 1}{(j\omega + 4)^2}
    \end{align*}
    
    By combining the part above;
    \begin{align*}
        Y(j\omega) = \frac{j\omega + 3}{(j\omega + 4)^2} \cdot \frac{j\omega + 4}{(j\omega + 2) \cdot (j\omega + 3)} \\
        Y(j\omega) = \frac{1}{(j\omega + 4) \cdot (j\omega + 2)}
    \end{align*}
    
    
    
    \item %write the solution of q2d
    \begin{align*}
        Y(j\omega) = \frac{1}{(j\omega + 4) \cdot (j\omega + 2)} = \frac{1}{2} \cdot [\frac{1}{(j\omega + 2)} - \frac{1}{(j\omega + 4)}]
    \end{align*}
    
    By applying inverse Fourier transform;
    \begin{align*}
        y(t) = \frac{1}{2} \cdot e^{-2t} \mu(t) - \frac{1}{2} \cdot e^{-4t} \mu(t) \\
        y(t) = \frac{1}{2} \cdot [e^{-2t} - e^{-4t}] \cdot \mu(t)
    \end{align*}
    
    \end{enumerate}

\item %write the solution of q3  
    \begin{enumerate}
    % Write your solutions in the following items.
    \item %write the solution of q3a
    \begin{align*}
        f(t) = e^{-|t|}\\
    \end{align*}
The Fourier transform of signal f(t):
    \begin{align*}
        F(\omega) = \int_{-\infty}^{\infty} f(t)e^{-j\omega t}dt\\
        F(\omega) = \int_{-\infty}^{\infty} e^{-|t|}e^{-j\omega t}dt\\
        F(\omega) = \int_{-\infty}^{0} e^{t}e^{-j\omega t}dt + \int_{0}^{\infty} e^{-t}e^{-j\omega t}dt\\
        F(\omega) = \int_{-\infty}^{0} e^{t(1-j\omega)}dt + \int_{0}^{\infty} e^{-t(1+j\omega)}dt\\
        F(\omega) = [\frac{1}{1-j\omega}e^{t(1-j\omega)}]_{-\infty}^{0} - [\frac{1}{(1+j\omega)}e^{-t(1+j\omega)}]_{0}^{\infty}\\
        F(\omega) = [\frac{1}{1-j\omega} - 0] -[0 - \frac{1}{1+j\omega}]\\
        F(\omega) = [\frac{1}{1-j\omega}] + [\frac{1}{1+j\omega}]\\
    \end{align*}
So the Fourier transform of  \(f(t) = e^{-|t|}\)
    \begin{align*}
        F(\omega) = \frac{2}{1+\omega^2}
    \end{align*}
    \item %write the solution of q3b
    Fourier transform of 
    \begin{align*}
        f(t) = e^{-|t|} \text{ is}\\
        F(\omega) = \frac{2}{1+\omega^2}\\
    \end{align*}
    Fourier transform of
    \begin{align*}
        x(t) = te^{-|t|}\\
        x(t) = tf(t) \text{ is}\\
    \end{align*}
By using multiplication of t with f(t) property;
    \begin{align*}
        X(\omega) = j\frac{dF(\omega)}{d\omega}\\
        X(\omega) = j[-\frac{4\omega}{(1+\omega^2)^2}]\\
    \end{align*}
So the Fourier Transform of \(x(t) = te^{-|t|}\)
    \begin{align*}
        X(\omega) = -j[\frac{4\omega}{(1+\omega^2)^2}]\\
    \end{align*}
    \item %write the solution of q3c
    Let;
    \begin{align*}
        g(t) = \frac{4t}{(1+t^2)^2}\\
    \end{align*}
The Fourier transform of \(x(t) = te^{-|t|}\) is
    \begin{align*}
        X(\omega) = -j[\frac{4\omega}{(1+\omega^2)^2}]\\
        x(t) \leftrightarrow X(\omega)\\
        te^{-|t|} \leftrightarrow -j[\frac{4\omega}{(1+\omega^2)^2}]\\
    \end{align*}
By using the duality property of Fourier transform;
    \begin{align*}
        \text{If } x(t) \leftrightarrow X(\omega) \text{ then }\\
        X(t) \leftrightarrow 2\pi x(-\omega)\\
        -j[\frac{4t}{(1+t^2)^2}] \leftrightarrow 2\pi(-\omega)e^{-|(-\omega)|}\\
        -j[\frac{4t}{(1+t^2)^2}] \leftrightarrow -2\pi(\omega)e^{-|\omega|}\\
        j[\frac{4t}{(1+t^2)^2}] \leftrightarrow 2\pi(\omega)e^{-|\omega|}\\
    \end{align*}
Dividing both sides by j;
    \begin{align*}
        \frac{4t}{(1+t^2)^2} \leftrightarrow \frac{1}{j}2\pi\omega e^{-|w|}\\
        \frac{4t}{(1+t^2)^2} \leftrightarrow -j\cdot 2\pi\omega e^{-|w|}\\
    \end{align*}
Hence, The Fourier Transform of 
    \begin{align*}
        g(t) = \frac{4t}{(1+t^2)^2} \text{ is }\\
        G(\omega) = -j\cdot2\pi\omega e^{-|\omega|}
        \end{align*}
    \end{enumerate}

\item %write the solution of q4
    \begin{enumerate}
    % Write your solutions in the following items.
    \item %write the solution of q4a
    The difference equation can be denoted as;
    \begin{align*}
        2 x(n) + \frac{3}{4} y(n-1) - \frac{1}{8} y(n-2) = y(n)
    \end{align*}
    
    \item %write the solution of q4b
    In order to find the frequency response of the system;
    \begin{align*}
        2X(e^j\omega) + \frac{3}{4} e^{-j\omega} Y(e^{j\omega}) - \frac{1}{8} e^{-2j\omega} Y(e^{j\omega}) = Y(e^{-j\omega}) \\
        Y(e^{-j\omega}) \cdot [1 - \frac{3}{4} e^{-j\omega} + \frac{1}{8} e^{-2j\omega}] = 2X(e^{j\omega}) \\
        \text{We know that ;} \\
        H(e^{j\omega}) = \frac{Y(e^{j\omega})}{X(e^{j\omega})} \\
        H(e^{j\omega}) = \frac{2}{1 - \frac{3}{4} e^{-j\omega} + \frac{1}{8} e^{-2j\omega}}
    \end{align*}
    
    \item %write the solution of q4c
    In order to find the impulse response of the system;
    \begin{align*}
        H(e^{j\omega}) = \frac{2}{(1 - \frac{1}{2} e^{-j\omega}) \cdot (1 - \frac{1}{4} e^{-j\omega})} \\
        H(e^{j\omega}) = \frac{A}{(1 - \frac{1}{2} e^{-j\omega})} + \frac{B}{(1 - \frac{1}{4} e^{-j\omega})} \\
        ... \\
        H(e^{j\omega}) = \frac{4}{(1 - \frac{1}{2} e^{-j\omega})} - \frac{2}{(1 - \frac{1}{4} e^{-j\omega})} 
    \end{align*}
    
    Therefore the answer becomes; 
    \begin{align*}
        f(n) = 4 (\frac{1}{2})^n \mu(n) - 2 (\frac{1}{4})^n \mu(n)
    \end{align*}
    
    \item %write the solution of q4d
    Given;
    \begin{align*}
        x(n) = (\frac{1}{4})^n \mu(n) \\
        x(n) = \frac{1}{1 - \frac{1}{4} e^{-j\omega}}
    \end{align*}
    
    So $Y(e^{j\omega})$ becomes;
    \begin{align*}
        Y(e^{j\omega}) = X(e^{j\omega}) \cdot H(e^{j\omega}) \\
        Y(e^{j\omega}) = \frac{2}{(1 - \frac{1}{2} e^{-j\omega}) \cdot ((1 - \frac{1}{4} e^{-j\omega})^2)}
    \end{align*}
    
    By applying the partial fraction expansion;
    \begin{align*}
        Y(e^{j\omega}) = \frac{A}{1 - \frac{1}{2}e^{-j\omega}} + \frac{B_0}{1 - \frac{1}{4}e^{-j\omega}} + \frac{B_1}{(1 - \frac{1}{4}e^{-j\omega})^2} \\
        ... \\
        B_1 = -2 \\
        B_0 = 4 \\
        A = 8
    \end{align*}
    
    So the equation becomes;
    \begin{align*}
        Y(e^{j\omega}) = \frac{8}{1 - \frac{1}{2}e^{-j\omega}} + \frac{4}{1 - \frac{1}{4}e^{-j\omega}} - \frac{2}{(1 - \frac{1}{4}e^{-j\omega})^2} \\
        y(n) = 8 (\frac{1}{2})^n \mu(n) + 4 (\frac{1}{4})^n \mu(n) - 2 (n+1) (\frac{1}{4})^n \mu(n) 
    \end{align*}
    
    \end{enumerate}

\item %write the solution of q5
When two systems are said to be connected in parallel the frequency response of the combined system becomes;
\begin{align*}
    H(e^{j\omega}) = H_{1}(e^{j\omega}) + H_{2}(e^{j\omega}) \\ 
    \text{So $H_1$ becomes,} \\
    H_{1}(e^{j\omega}) = \frac{1}{1 - \frac{1}{3}e^{-j\omega}}
\end{align*}

So the we can write the equation as,
\begin{align*}
    \frac{5e^{-j\omega} - 12}{e^{-2j\omega} - 7e^{-j\omega} + 12} = \frac{1}{1 - \frac{1}{3}e^{-j\omega}} + H_{2}(e^{j\omega})
\end{align*}

So to find $H_2$;
\begin{align*}
    \frac{\frac{5}{12} e^{-j\omega} - 1}{(1 - \frac{1}{3}e^{-j\omega}) \cdot (1 - \frac{1}{4}e^{-j\omega})} = \frac{1}{1 - \frac{1}{3}e^{-j\omega}} + H_{2}(e^{j\omega}) \\
    \frac{\frac{5}{12} e^{-j\omega} - 1}{(1 - \frac{1}{3}e^{-j\omega}) \cdot (1 - \frac{1}{4}e^{-j\omega})} = \frac{1}{1 - \frac{1}{3}e^{-j\omega}} + \frac{A}{1 - \frac{1}{4}e^{-j\omega}}
\end{align*}

So we can deduct that $H_{2}(e^{j\omega}) = \frac{A}{1 - \frac{1}{4}e^{-j\omega}}$. \\
So by solving the mathematical equations, we were left with this;
\begin{align*}
    \frac{5}{12} e^{-j\omega} - 1 = (1 + A) + (- \frac{1}{4} - \frac{A}{3}) e^{-j\omega} \\
    \text{by equating the $(e^{-j\omega})$ terms,} \\
    \frac{5}{12} = -\frac{1}{4} - \frac{A}{3} \\
    A = -2 \\
\end{align*}

So the answer becomes,
\begin{align*}
    h_{2}(n) = -2 (\frac{1}{4})^n \mu(n)
\end{align*}

\item %write the solution of q6
    \begin{enumerate}
    % Write your solutions in the following items.
    \item %write the solution of q6a
    To determine the difference equation, we represent $H(e^{j\omega})$ as $\frac{Ye^{j\omega}}{X(e^{j\omega})}$.
    \begin{align*}
        H(e^{j\omega}) = \frac{Y(e^{j\omega})}{X(e^{j\omega})} = \frac{1}{1 - \frac{1}{6}e^{-j\omega} - \frac{1}{6}e^{-j2\omega}}\\
        Y(e^{j\omega}) - \frac{1}{6}e^{-j\omega}Y(e^{j\omega}) - \frac{1}{6}e^{-j2\omega}Y(e^{j\omega}) = X(e^{j\omega})\\
    \end{align*}
By applying Inverse Discrete-Time Fourier transform on both sides;
    \begin{align*}
        y(n) - \frac{1}{6}y(n-1) - \frac{1}{6}y(n-2) = x(n)\\
    \end{align*}
So the difference equation is
    \begin{align*}
        y(n) - \frac{1}{6}y(n-1) - \frac{1}{6}y(n-2) = x(n)
    \end{align*}
    
    
    \item %write the solution of q6b
    The difference equation by part a;\\
    \begin{align*}
        y(n) - \frac{1}{6}y(n-1) - \frac{1}{6}y(n-2) = x(n)
    \end{align*}
    By comparing with linear constant coefficient difference equation;
    \begin{align*}
        a_0 y(n) + a_1 y(n-1) + a_2 y(n-2)+...+a_N y(n-N) = b_0 x(n) + b_1 x(n-1)+...+b_M X(n-M)
    \end{align*}
    Here \(a_0 = 1, a_1 = \frac{-1}{6}, a_2 = \frac{-1}{6}, b_0 = 1\)\\
    So the block diagram by using this form;
    \\
\begin{center}
    
    \tikzset{%
    		block/.style    = {draw, thick, rectangle, minimum height = 3em,
    			minimum width = 3em},
    		sum/.style      = {draw, circle, node distance = 2cm}, % Adder
    		sum/.style      = {draw, circle, node distance = 2cm}, % Adder
    		input/.style    = {coordinate}, % Input
    		output/.style   = {coordinate} % Output
    	}
    	% Defining string as labels of certain blocks.
    	\newcommand{\suma}{\Large$+$}
    	\newcommand{\sumb}{\Large$+$}
    	\newcommand{\inte}{$\displaystyle Z^{-1}$}
    	\newcommand{\derv}{\huge$\frac{d}{dt}$}
    	
    	\begin{tikzpicture}[auto, thick, node distance=3cm, >=triangle 45]
    	\draw
    	% Drawing the blocks of first filter :
    	node at (0, 0) [input] (inp) {\Large \textopenbullet}
    	node [sum, right of=inp] (suma) {\suma}
    	node [sum, below of=suma] (sumb) {\sumb}
    	node [block, right of=sumb] (inta) {$Z^{-1}$}
    	node [block, below of=inta] (intb) {$Z^{-1}$}
    	node [output, right of=suma] (out) {}
    	node [output, right of=out] (out2) {}
    	node [output, right of=out2] (out3) {}
    	node [output, below of=out] (temp1) {}
    	node [output, below of=suma] (temp2) {}
    	node [output, below of=intb] (temp3) {}
    	node [output, below of=sumb] (temp4) {}
    	;
    	\draw[->](inp) -- node{$x(n)$} (suma);
    	\draw[->](suma) -- node{$a_0 = 1$}(out);
    	\draw[->](out) -- node{$b_0 = 1$} (out2);
    	\draw[-](out) -- (inta);
    	\draw[.](out2) -- node{$y(n)$}(out3);
    	\draw[->](sumb) -- (suma);
    	\draw[-](inta) -- (intb);
    	\draw[->](inta) -- node{$-a_1 = \frac{1}{6}$} (sumb);
    	\draw[->](temp4) -- (sumb);
    	\draw[-](intb) -- node{$-a_2 = \frac{1}{6}$} (temp4);
	\end{tikzpicture}
    
    
\end{center}
    \item %write the solution of q6c
    Given;
    \begin{align*}
        H(e^{j\omega}) = \frac{1}{1 - \frac{1}{6}e^{-j\omega} - \frac{1}{6}e^{-j2\omega}}\\
        H(e^{j\omega}) = \frac{6}{6 - e^{-j\omega} - e^{-j2\omega}}\\
        H(e^{j\omega}) = \frac{6}{6 - 3e^{-j\omega} + 2e^{-j\omega} -e^{-j2\omega}}\\
        H(e^{j\omega}) = \frac{6}{3(2- e^{-j\omega}) + e^{-j\omega}(2 - e^{-j\omega})}\\
        H(e^{j\omega}) = \frac{6}{(3+e^{-j\omega})(2-e^{-j\omega})}\\
    \end{align*}
    
By applying Partial Fraction; 
    \begin{align*}
        \frac{6}{(3+e^{-j\omega})(2-e^{-j\omega})} = \frac{A}{3+e^{-j\omega}} + \frac{B}{2-e^{-j\omega}}\\
        A= \frac{6}{5}\\
        B= \frac{6}{5}\\
        H(e^{j\omega}) = \frac{\frac{6}{5}}{3+e^{-j\omega}} + \frac{\frac{6}{5}}{2-e^{-j\omega}}\\
        H(e^{j\omega}) = \frac{\frac{6}{5}}{3(1+\frac{1}{3}e^{-j\omega})} + \frac{\frac{6}{5}}{2(1-\frac{1}{2}e^{-j\omega})}\\
        H(e^{j\omega}) = \frac{\frac{2}{5}}{1+\frac{1}{3}e^{-j\omega}} + \frac{\frac{3}{5}}{1-\frac{1}{2}e^{-j\omega}}\\
    \end{align*}
By applying inverse Discrete-Time Fourier transform on both sides;
    \begin{align*}
        h(n) = \frac{2}{5} \cdot (\frac{-1}{3})^{n}u(n) + \frac{3}{5} \cdot (\frac{1}{2}^{n})u(n).\\
    \end{align*}
    So Impulse Response is;
    \begin{align*}
        h(n) = \frac{2}{5} \cdot (\frac{-1}{3})^{n}u(n) + \frac{3}{5} \cdot (\frac{1}{2}^{n})u(n).
    \end{align*}
    \end{enumerate}    

\end{enumerate}
\end{document}

