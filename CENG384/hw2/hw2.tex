\documentclass[10pt,a4paper, margin=1in]{article}
\usepackage{fullpage}
\usepackage{amsfonts, amsmath, pifont}
\usepackage{amsthm}
\usepackage{graphicx}
\usepackage{tkz-euclide}
\usepackage{tikz}
\usepackage{pgfplots}

\usepackage{geometry}
 \geometry{
 a4paper,
 total={210mm,297mm},
 left=10mm,
 right=10mm,
 top=10mm,
 bottom=10mm,
 }
 % Write both of your names here. Fill exxxxxxx with your ceng mail address.
 \author{
  Karaman, Arda\\
  \texttt{e2237568@ceng.metu.edu.tr}
  \and
  Fulser, Göktürk\\
  \texttt{e2237386@ceng.metu.edu.tr}
}
\title{CENG 384 - Signals and Systems for Computer Engineers \\
Spring 2021 \\
Homework 2}

\begin{filecontents}{q1.dat}
 n   xn
 0   0
 1   1  
 2   -1
 3   1
 4   -1 
 5   1
 6   -1
 7   1
\end{filecontents}

\begin{filecontents}{q2.dat}
 n   xn
-3   6
-2   0
-1   0
 0   2
 1   3  
 2   0
 3   0
 4   1 
\end{filecontents}

\begin{filecontents}{q3.dat}
 n   xn
 -2   1
 -1   2 
 0   2
 1   1  
 2   0
 3   0
 4   0 
\end{filecontents}

\begin{filecontents}{q7.dat}
 n   xn
-1  0
 0   0  
 1   0
 2   0
 3   1 
 4   0
 5  -1
 6   0
\end{filecontents}

\begin{document}
\maketitle



\noindent\rule{19cm}{1.2pt}

\begin{enumerate}

\item %write the solution of q1
    \begin{enumerate}
    % Write your solutions in the following items.
    \item %write the solution of q1a
    \begin{align}
        \int x(t) - 5 \cdot y(t) - (\int 6 \cdot y(t)dt) \cdot dt = y(t) \\
        y^{\prime\prime}(t) + 5 \cdot y^{\prime}(t) + 6 \cdot y(t) = x^{\prime}(t)
    \end{align}
    \item %write the solution of q1b
    The complete solution consists two parts; a homogeneous and a particular solution. \\
    \begin{align}
        y(t) = y_{h}(t) + y_{p}(t)
    \end{align}
    The system is initially at rest and with the homogeneous solution;
    \begin{align}
        y_{h}(t) = K e^{\alpha t} \\
        (\alpha^{2} + 5 \alpha + 6) \cdot Ke^{\alpha t} = 0 \\
        (\alpha + 2) \cdot (\alpha + 3) = 0 \\
        \alpha_{1,2} = -2,-3
    \end{align}
    Therefore there are 2 solutions for the homogeneous part of the system.
    For the particular solution:
    \begin{align}
        x(t) = (e^{-t} + e^{-4t})u(t) \\
        y_{p}(t) = \alpha e^{-t} + \beta e^{-4t} \\
        \alpha e^{-t} + 16\beta e^{-4t} + 5 (-\alpha e^{-t} - 16\beta e^{-4t}) + 6 (\alpha e^{-t} + 16\beta e^{-4t}) =  (e^{-t} + e^{-4t})^{\prime} \\
        2 \alpha e^{-t} + 2\beta e^{-4t} = -e^{-t} -4 \cdot e^{-4t} \\ 
        \alpha = -\frac{1}{2} , \beta = -2
    \end{align}
    Summing up the y(t):
    \begin{align}
        y(t) = y_{h}(t) + y_{p}(t) \\
        y(t) = Ke^{-2t} - \frac{1}{2} e^{-t} - 2 \cdot e^{-4t} \\
    \end{align}
    Since the system is initially at rest;
    \begin{align}
        y(0) = 0 \\
        0 = K - \frac{1}{2} - 2 \\
        \frac{5}{2} =  K
    \end{align}
    Therefore the final result is,
    \begin{align}
        y(t) = (-\frac{1}{2} e^{-t} - 2\cdot e^{-4t} + \frac{5}{2} \cdot e^{-2t})u(t)
    \end{align}
    
    \end{enumerate}

\item %write the solution of q2
    \begin{enumerate}
    % Write your solutions in the following items.
    \item %write the solution of q2a
    x(n) can be written as :
    \begin{align*}
        x(n) = \delta(n) + \delta(n-1) \\
        y(n) = \delta(n-1)
    \end{align*}
    
    $x^{\prime} $ can be written w.r.t. x(n) as
    \begin{align*}
        x^{\prime} = x(n) - x(n-2)
    \end{align*}
    
    By the linearity theorem:
    \begin{align*}
        x(n) \longrightarrow y(n) \\
        x(n-2) \longrightarrow y(n-2) 
    \end{align*}
    Therefore,
    \begin{align*}
        y^{\prime}(n) = y(n) - y(n-2) \\
        = \delta(n-1) - \delta(n-3)
    \end{align*}
    
    \item %write the solution of q2b
    
    
    
    \begin{align*}
        Y(Z) = X(Z) \cdot H(Z) \\
        H(Z) = \frac{Y(Z)}{X(Z)} \\ 
        x(n) = \delta(n) + \delta(n+1) \Longrightarrow X(Z) = 1 + Z^{-1} = \frac{1+ Z}{Z} \\
        y(n) = \delta(n-1) \Longrightarrow y(Z) = Z^{-1} = \frac{1}{Z} \\
        H(Z) = \frac{\frac{1}{Z}}{\frac{Z+1}{Z}} = \frac{1}{Z+1} = \frac{Z^{-1}}{1+ Z^{-1}}
    \end{align*}
    The impulse response, \\
    \begin{figure} [h!]
        \centering
        \begin{tikzpicture}[scale=1.0] 
          \begin{axis}[
              axis lines=middle,
              xlabel={$n$},
              ylabel={$\boldsymbol{h[n]}$},
              xtick={ -1, 0,  ..., 8},
              ytick={ -2, -1, ..., 2},
              ymin=-2, ymax=2,
              xmin=-1, xmax=8,
              every axis x label/.style={at={(ticklabel* cs:1.05)}, anchor=west,},
              every axis y label/.style={at={(ticklabel* cs:1.05)}, anchor=south,},
              grid,
            ]
            \addplot [ycomb, black, thick, mark=*] table [x={n}, y={xn}] {q1.dat};
          \end{axis}
        \end{tikzpicture}
        \caption{$n$ vs. $h[n]$.}
        \label{fig:q2}
    \end{figure}
    
    \item %write the solution of q2c
    In the last part we figured out that,
    \begin{align*}
        H(Z) = \frac{1}{Z+1} \\
        \frac{Y(Z)}{X(Z)} = \frac{Z^{-1}}{1+ Z^{-1}}
    \end{align*}
    So that, 
    \begin{align*}
        (1 + Z^{-1}) \cdot Y(Z) = Z^{-1} \cdot X(Z) \Longrightarrow X(Z) + Z^{-1} \cdot Y(Z) = Z^{-1} \cdot X(Z)
    \end{align*}
    And by taking the inverse z-transform of the equation above,
    \begin{align}
        y(n) + y(n-1) = x(n-1)
    \end{align}
    
    \item %write the solution of q2d
    We have already got the equation, 
    \begin{align}
        Y(Z) = \frac{Z^{-1}}{1 + Z^{-1}} \cdot X(Z)
    \end{align}
    from the last part.
    So the block diagram of the total system is, \\
    
    \begin{center}
    	\tikzset{%
    		block/.style    = {draw, thick, rectangle, minimum height = 3em,
    			minimum width = 3em},
    		sum/.style      = {draw, circle, node distance = 2cm}, % Adder
    		input/.style    = {coordinate}, % Input
    		output/.style   = {coordinate} % Output
    	}
    	% Defining string as labels of certain blocks.
    	\newcommand{\suma}{\Large$+X$}
    	\newcommand{\inte}{$\displaystyle Z^{-1}$}
    	\newcommand{\derv}{\huge$\frac{d}{dt}$}
    	
    	\begin{tikzpicture}[auto, thick, node distance=2cm, >=triangle 45]
    	\draw
    	% Drawing the blocks of first filter :
    	node at (0, 0) [input] (inp) {\Large \textopenbullet}
    	node [sum, right of=inp] (sum) {\suma}
    	node [block, right of=sum] (int) {\inte}
    	node [output, right of=int] (out) {}
    	node [output, right of=out] (out2) {\Large \textopenbullet}
    	node [output, below of=out] (temp1) {}
    	node [output, below of=sum] (temp2) {}
    	;
    	\draw[->](inp) -- node{$X(Z)$} (sum);
    	\draw[->](sum) -- (int);
    	\draw[-](int) -- (out);
    	\draw[->](out) -- node{$Y(Z)$} (out2);
    	\draw[-](out) -- (temp1);
    	\draw[->](temp2) -- (sum);
    	\draw[-](temp1) -- node{$-H(Z)$} (temp2);
	\end{tikzpicture}
\end{center}
    
    
    \end{enumerate}

\item %write the solution of q3  
    \begin{enumerate}
    % Write your solutions in the following items.
    \item %write the solution of q3a
    
    \(x[n] = \delta[n-3] + 2\delta[n+1], \)\\
    \(h[n] = \delta[n-1] + 3\delta[n+2], \)\\
    To use \(h[n]\)  or \(x[n]\)  we have to divide one of them.
    Therefore;\\
    We divide \(h[n]\) into two sub-parts using the distribution property of convolution operator. \\
     \begin{align}
        (h[n] = h_a[n] + h_b[n]
    \end{align}
    \begin{align}
        h_a[n] =3\delta[n+2] , h_b[n] = \delta[n-1]
    \end{align}
    So \(y[n]\) becomes \(y[n] = x[n] \ast h_a[n] + x[n] \ast h_b[n]\)\\
    \(y_a[n] = (\delta[n-3] + 2\delta[n+1]) \ast 3\delta[n+2] = 3\delta[n-1] + 6\delta[n+3] \)\\
    \(y_b[n] = (\delta[n-3] + 2\delta[n+1]) \ast \delta[n-1] = \delta[n-4] + 2\delta[n] \)\\
    So \(y[n] =  3\delta[n-1] + 6\delta[n+3] + \delta[n-4] + 2\delta[n].\)\\
    \begin{figure} [h!]
        \centering
        \begin{tikzpicture}[scale=1.0] 
          \begin{axis}[
              axis lines=middle,
              xlabel={$n$},
              ylabel={$\boldsymbol{y[n]}$},
              xtick={ -3, -2, -1, 0,  ..., 4},
              ytick={-1, 0,  ..., 6},
              ymin=-1, ymax=6,
              xmin=-3, xmax=4,
              every axis x label/.style={at={(ticklabel* cs:1.05)}, anchor=west,},
              every axis y label/.style={at={(ticklabel* cs:1.05)}, anchor=south,},
              grid,
            ]
            \addplot [ycomb, black, thick, mark=*] table [x={n}, y={xn}] {q2.dat};
          \end{axis}
        \end{tikzpicture}
        \caption{$n$ vs. $y[n]$.}
        \label{fig:q3}
    \end{figure}
    
    
    \item %write the solution of q3b
    \(h[n] = u[n-1] - u[n-3], \)\\
    We can write \(h[n]\) in unit impulse form as follows: 
    \begin{align}
        h[n] = \delta[n-1] + \delta[n-2] \\
        x[n] = u[n+3] + u[n] 
    \end{align}
    We can write \(x[n]\) in unit impulse form as follows:
    \begin{align}
        x[n] = \delta[n+3] + \delta[n+2] + \delta[n+1]
    \end{align}
    \(h[n] = h_a[n] + h_b[n]\) by distribution property \\
    So \(h_a[n] = \delta[n-1], h_b[n] = 3\delta[n-2] \ and \ y[n] = x[n] \ast h_a[n] + x[n] \ast h_b[n]\)\\
    \begin{align*}
        y_a[n] = (\delta[n+3] + \delta[n+2] + \delta[n+1]) \ast \delta[n-1] = \delta[n+2] + \delta[n+1] + \delta[n]. \\
        y_b[n] = (\delta[n+3] + \delta[n+2] + \delta[n+1]) \ast \delta[n-2] = \delta[n+1] + \delta[n] + \delta[n-1].
    \end{align*}
    So \(y[n] = \delta[n+2] + 2\delta[n+1]+ 2\delta[n] + \delta[n-1].\)\\
    
    The figure is above.
    
    \begin{figure} [ht!]
        \centering
        \begin{tikzpicture}[scale=1.0] 
          \begin{axis}[
              axis lines=middle,
              xlabel={$n$},
              ylabel={$\boldsymbol{y[n]}$},
              xtick={-4 ,-3, -2, -1,  ..., 5},
              ytick={-1, 0,  ..., 4},
              ymin=-1, ymax=4,
              xmin=-4, xmax=5,
              every axis x label/.style={at={(ticklabel* cs:1.05)}, anchor=west,},
              every axis y label/.style={at={(ticklabel* cs:1.05)}, anchor=south,},
              grid,
            ]
            \addplot [ycomb, black, thick, mark=*] table [x={n}, y={xn}] {q3.dat};
          \end{axis}
        \end{tikzpicture}
        \caption{$n$ vs. $y[n]$. from question 3b}
        \label{fig:q3}
    \end{figure}
    
    
    \end{enumerate}

\item %write the solution of q4
    \begin{enumerate}
    % Write your solutions in the following items.
    \item %write the solution of q4a
    \(h(t) = e^{-3t}u(t) ,\ x(t) = e^{-2t}u(t) \\ y(t) = x(t) \ast h(t) \) \\
    \(y(t) =  \int_{0}^{t} (e^{-2\tau} e^{-3(t-\tau)} d\tau) = \int_{0}^{t} (e^{-2\tau} e^{-3t} e^{3\tau} d\tau) \) \\
    \(y(t) = e^{-3t}\int_{0}^{t}(e^\tau d\tau)=e^{-3t}(e^{t}-1) \)\\
    So \(y(t)=(e^{-2t}-e^{-3t})u(t)\)
    \item %write the solution of q4b
    \(h(t) = e^{-2t}u(t), \ x(t) = u(t) - u(t-2)\)\\
    \(y(t) = x(t) \ast h(t) \)\\
    \(y(t) = \int_{0}^{t}x(t)h(t-\tau)d\tau\)\\
    \(y(t) = u(t)\int_{0}^{t}(e^{-2(t-\tau)} d\tau)-
    u(t-2)\int_{0}^{t}(e^{-2(t-\tau)} d\tau)\)\\
    \(y(t) = u(t)(e^{-2t}(e^{2t}-1))-u(t-2)(e^{-2t}(e^{2t}-1))\)\\
    So \(y(t) = (1-e^{-2t})\cdot (u(t)-u(t-2))\)
    \end{enumerate}

\item %write the solution of q5
    \begin{enumerate}
    % Write your solutions in the following items.
    \item %write the solution of q5a
    \begin{align}
        s[n] = nu[n] \\
        h[n] = nu[n]-(n-1)u[n-1] = u[n-1]
    \end{align}

    \item %write the solution of q5b
    \begin{align}
        y[n] - y[n-1] = x[n] \ast (h[n] - h[n-1]) \\
        x[n-1] = \delta[n] - 2\delta[n-1] + \delta[n-2] = x[n] \ast \delta[n-1] \\
        x[n] = \delta[n+1] - 2\delta[n] + \delta[n-1]
    \end{align}
    
    \item %write the solution of q5c
    \begin{align}
        y[n] = y[n+1] - x[n]
    \end{align}
    
    \end{enumerate}

\item %write the solution of q6
\(s(t) = \frac{1}{2}t^{2}u(t).\)\\
\(h(t) = \frac{ds(t)}{dt} = tu(t),x(t) = e^{-t}u(t) .\)\\
\(y(t) = h(t) \ast x(t)\)\\
\(h(t-\tau) = (t-\tau)u(t-\tau)\)\\
\(y(t) = \int_{0}^{t}e^{-\tau}(t-\tau)d\tau\)\\
\(y(t) = t\int_{0}^{t}e^{-\tau}d\tau - \int_{0}^{t}e^{-\tau}\tau d\tau \)\\
\(y(t) = t(1-e^{-t}) + te^{-t} + e^{-t} -1\)\\
So \(y(t) = t + e^{-t} -1.\)
\item %write the solution of q7
    \begin{enumerate}
    % Write your solutions in the following items.
    \item %write the solution of q7a
    \(x(t) = u(t) \)\\
    \(y(t) = u(t-3) - u(t-5)\)\\
    \(h(t) =  \delta(t-3)-\delta(t-5)\)\\ \\
    The figure is below.
    \begin{figure} [ht!]
        \centering
        \begin{tikzpicture}[scale=1.0] 
          \begin{axis}[
              axis lines=middle,
              xlabel={$n$},
              ylabel={$\boldsymbol{h[n]}$},
              xtick={-1,  ..., 6},
              ytick={-2, -1, 0,  ..., 2},
              ymin=-2, ymax=2,
              xmin=-1, xmax=6,
              every axis x label/.style={at={(ticklabel* cs:1.05)}, anchor=west,},
              every axis y label/.style={at={(ticklabel* cs:1.05)}, anchor=south,},
              grid,
            ]
            \addplot [ycomb, black, thick, mark=*] table [x={n}, y={xn}] {q7.dat};
          \end{axis}
        \end{tikzpicture}
        \caption{$n$ vs. $h[n]$(continuous impulse response) from question 7a}
        \label{fig:q7}
    \end{figure}
    \item %write the solution of q7b
    \(x(t) = e^{-3t}u(t)\)\\
    \(x(t-\tau) = e^{-3(t-\tau)}\)\\
    \(y(t) = \int_{0}^{t}x(t-\tau)h(\tau)d\tau\)\\
    \(y(t) = \int_{0}^{t}e^{-3(t-\tau)}d\tau\)\\
    \(y(t) = e^{-3t}(e^{3t}-1)\)\\
    So \(y(t) = (1-e^{-3t})(\delta(t-3)-\delta(t-5))\)
    \item %write the solution of q7c
    \end{enumerate}    

\end{enumerate}
\end{document}

