\documentclass[10pt,a4paper, margin=1in]{article}
\usepackage{fullpage}
\usepackage{amsfonts, amsmath, pifont}
\usepackage{amsthm}
\usepackage{graphicx}
\usepackage{float}
\usepackage{tkz-euclide}
\usepackage{tikz}
\usepackage{pgfplots}
\pgfplotsset{compat=1.13}

\usepackage{geometry}
 \geometry{
 a4paper,
 total={210mm,297mm},
 left=10mm,
 right=10mm,
 top=10mm,
 bottom=10mm,
 }
 
 % Write both of your names here. Fill exxxxxxx with your ceng mail address.
 \author{
  Arda, Karaman\\
  \texttt{e2237568@ceng.metu.edu.tr}
  \and
  Göktürk, Fulser\\
  \texttt{e2237386@ceng.metu.edu.tr}
}
\title{CENG 384 - Signals and Systems for Computer Engineers \\
Spring 2021 \\
Homework 1}

\begin{filecontents}{q4.dat}
  n  xn
 -7  -4
 -6  0  
 -5  0
 -4  3
 -3  0
 -2  -2
 -1  1
  0  0
  1  1
  2  0
  3  -4
\end{filecontents}

\begin{filecontents}{q7.dat}
  n  xn
  -1 0
  0  0
  1  0
  2  -3
  3  5
  4  0
  5  -3
  6  0
\end{filecontents}

\begin{document}
\maketitle



\noindent\rule{19cm}{1.2pt}

\begin{enumerate}

\item %write the solution of q1
Proof of: \( \frac{ de^t }{ dt } = e^t \)

\( [e^t]' = \lim_{h \to 0} \frac{e^{t+h} - e^t}{h} \) \\
\( [e^t]' = e^t \cdot \lim_{h \to 0} \frac{e^{h} - 1}{h} \) \\ 
From the definition of Euler number e is the only number for: \( \lim_{h \to 0} \frac{e^{h} - 1}{h} = 1 \) \\
Therefore \( [e^t]' = e^t \cdot 1\) \\


\item %write the solution of q2
    \begin{enumerate}
    % Write your solutions in the following items.
    \item %write the solution of q2a
    i.\\
    \(z=x+jy,\ \overline{z}=x-jy,\ z-3=j-2\overline{z} \)\\
    \(x+jy-3=j-2(x-jy)\)\\
    %%%%%%%%%%%%%%%%%%%%%%%%%%%%%%%%%%%%%%%%%%%%%%%%%%%%%%%%%%%%%%%%%%%%%%%%%%%%
    \(3(x-1)=yj+j\)\\
    \(3x-jy=j+3\ so\ x=1, y=-1. \)\\
    Since \(|z|^2=x^2+y^2\ \)and \(z=1-j\)\\
    \(|z|^2=2\)\\
    ii.\\
    \begin{figure}[h!]
    \centering
        \begin{tikzpicture}[scale=1.0]
           \begin{axis}[
          axis lines=middle,
          xlabel={$x$},
          ylabel={$y$},
          xtick={-2, -1, ..., 2},
          ytick={-2, -1, ..., 2},
          ymin=-2, ymax=2,
          xmin=-2, xmax=2,
          every axis x label/.style={at={(ticklabel* cs:1.05)}, anchor=northwest,},
          every axis y label/.style={at={(ticklabel* cs:1.05)}, anchor=south,},
          grid,
        ]
           \path[draw,line width=4pt] (0,0) -- (1,-1);
           \end{axis}
        \end{tikzpicture}
        \caption{ $z$ on complex plane.}
        \label{fig:q2}
    \end{figure}
    \item %write the solution of q2b
    \(z^4=-81\ z^2=9i\ or\ z^2=-9i\)\\
    If \(z^2=9i\ z=\frac{3\sqrt{2}}{2}+\frac{3\sqrt{2}}{2}i\ or\ \frac{-3\sqrt{2}}{2}+\frac{-3\sqrt{2}}{2}i\);\\
    it's polar form = \(re^{j\theta}\)\\
    \(r=\sqrt{(\frac{3\sqrt{2}}{2})^2+(\frac{3\sqrt{2}}{2})^2}\ or\ \sqrt{(\frac{-3\sqrt{2}}{2})^2+(\frac{-3\sqrt{2}}{2})^2}\) both equals \(r=3\)\\
    \(\theta=\tan^{-1}(\frac{\frac{3\sqrt{2}}{2}}{\frac{3\sqrt{2}}{2}})\ or\ 
    \tan^{-1}(\frac{\frac{-3\sqrt{2}}{2}}{\frac{-3\sqrt{2}}{2}})\) both equals \(\theta=\frac{\pi}{4}\)\\
    So it's polar form is \(3e^{j\frac{\pi}{4}}\).\\
    If \(z^2=-9i\ z=\frac{-3\sqrt{2}}{2}+\frac{3\sqrt{2}}{2}i\ or\ \frac{3\sqrt{2}}{2}+\frac{-3\sqrt{2}}{2}i\);\\
    it's polar form = \(re^{j\theta}\)\\
    \(r=\sqrt{\frac{-3\sqrt{2}}{2}+\frac{3\sqrt{2}}{2}}\ or\ \sqrt{\frac{3\sqrt{2}}{2}+\frac{-3\sqrt{2}}{2}}\) both equals \(r=3\)\\
    \(\theta=\tan^{-1}(\frac{\frac{-3\sqrt{2}}{2}}{\frac{3\sqrt{2}}{2}})\ or\ 
    \tan^{-1}(\frac{\frac{3\sqrt{2}}{2}}{\frac{-3\sqrt{2}}{2}})\) both equals \(\theta=\frac{3\pi}{4}\)\\
    So it's polar form is \(3e^{j\frac{3\pi}{4}}\).
    \item %write the solution of q2c
    Magnitude = \(\frac{\sqrt{(\frac{1}{2})^2 + (\frac{1}{2})^2)} \cdot \sqrt{(1)^2 + (-1)^2}}{\sqrt{(1)^2 + (-\sqrt{3})^2}}=\frac{1}{2}\)\\ Angle = \(\angle (\frac{(\frac{1}{2} + \frac{1}{2}j) \cdot (1 - j)}{(1-\sqrt{3}j)}) = \angle (\frac{1}{2}+\frac{1}{2}j)+\angle (1-j) - \angle (1-\sqrt{3}j) = \frac{\pi}{3} \)
    \item %write the solution of q2d
    \(z= \frac{-3}{j}\cdot e^{j\frac{\pi}{2}}\)\\
    \(e^{j\frac{\pi}{2}}=j\) so \(z=\frac{-3j}{j}\)\\
    \(z=-3\)\\
    It's polar form = \(3e^{j\pi}\)
    \end{enumerate}

\item %write the solution of q3    
Below is the signal for $y(t) = 2 \cdot x(\frac{1}{2}t + 3)$
\begin{figure}[h!]
    \centering
        \begin{tikzpicture}[scale=1.0]
           \begin{axis}[
          axis lines=middle,
          xlabel={$t$},
          ylabel={$\boldsymbol{y(t)}$},
          xtick={-11, -10, -9, -8, ..., 0},
          ytick={-3, -2, -1, ..., 3},
          ymin=-3, ymax=3,
          xmin=-11, xmax=1,
          every axis x label/.style={at={(ticklabel* cs:1.05)}, anchor=northwest,},
          every axis y label/.style={at={(ticklabel* cs:1.05)}, anchor=south,},
          grid,
        ]
           \path[draw,line width=4pt] (-11,0) -- (-10,0)  -- (-8,2) -- (-4,2) -- (-4,0) -- (0,0) -- (4,0);
           \end{axis}
        \end{tikzpicture}
        \caption{$t$ vs. $x(t)$.}
        \label{fig:q2}
    \end{figure}

\item %write the solution of q4
    \begin{enumerate}
    % Write your solutions in the following items.
    \item %write the solution of q4a
        Below is the signal for $x[-n] + x[2n + 1]$
        \begin{figure}[H]
            \centering
            \begin{tikzpicture}[scale=1.0] 
              \begin{axis}[
                  axis lines=middle,
                  xlabel={$n$},
                  ylabel={$\boldsymbol{x[-n]+x[2n+1]}$},
                  xtick={ -9, -8, ..., 4},
                  ytick={-4, -3, -2, -1, ..., 4},
                  ymin=-4, ymax=4,
                  xmin=-8, xmax=4,
                  every axis x label/.style={at={(ticklabel* cs:1.05)}, anchor=west,},
                  every axis y label/.style={at={(ticklabel* cs:1.05)}, anchor=south,},
                  grid,
                ]
                \addplot [ycomb, black, thick, mark=*] table [x={n}, y={xn}] {q4.dat};
              \end{axis}
            \end{tikzpicture}
            \caption{$n$ vs. $x[-n]+x[2n+1]$.}
            \label{fig:q3}
        \end{figure}
        
        
    \item %write the solution of q4b
    
    $x[-n] + x[2n+1]$ in terms of the unit impulse function is:\\
    
    $-4\delta(n+7)+3\delta(n+4)-2\delta(n+2)+\delta(n+1)+\delta(n-1)-4\delta(n-3)$ \\
    
    \end{enumerate}

\item %write the solution of q5
    \begin{enumerate}
    % Write your solutions in the following items.
    \item %write the solution of q5a
    \( x(t) =3\cos(7\pi t-\frac{4\pi}{5})=x(t+T) \) \\
    \(\omega=\frac{2\pi}{T}\) \\
    \(\omega=7\pi\) \\
    \(7\pi=\frac{2\pi}{T}\) \\
    \(T=\frac{2}{7}\) \\
    Since T equals a finite and nonzero real value, \(x(t)\) is periodic and \
    \(T=\frac{2}{7}\).
    \item %write the solution of q5b
    \(x[n]=\sin[4n-\frac{\pi}{2}]=x[n+N_0]\)\\
    \(\Omega=\frac{2\pi}{N_0}\)\\
    \(\Omega=4\)\\
    \(4N_0=2\pi m\ \)where m is an integer.\\
    \(N_0=\frac{\pi m}{2}\)\\
    There is no integer that satisfies this equation so it is not periodic.
    \item %write the solution of q5c
    \(x[n]=2\cos[\frac{7\pi}{5}n]+ 7\sin[\frac{5\pi}{2}n-\frac{\pi}{3}]\)\\
    \(\frac{7\pi}{5}N_1=2\pi m_1\quad \frac{5\pi}{2}N_2=2\pi m_2 \)\\
    \(N_1=\frac{10m_1}{7} \quad N_2=\frac{4m_2}{5} \)\\
    \(N_1=10 \quad N_2=4\)\\
    Least Common multiplier of \(N_1\) and \(N_2\) is 20. So period of the signal is 20.
    \end{enumerate}


\item %write the solution of q6
    \begin{enumerate}
    % Write your solutions in the following items.
    \item 
    Functions whose graphs are symmetric about the y-axis are called even functions. \\
    \( x(t) \neq x(-t) \) \\
    Functions with a graph that is symmetric about the origin is called an odd function. \\
    \( x(t) \neq -x(-t) \) \\
    Since \( x(t) \) does not satisfy these conditions, signal \( x(t) \) is neither odd nor even.
    \item %write the solution of q6b
    We've already found whether the signal is even or odd. Now, in order to find even and odd decompositions of $x(t)$, we have:
        \begin{align*}
        x(t)&=\text{Ev\{x(t)\}} + \text{Odd\{x(t)\}} \\
        x(t)&=\frac{1}{2}\{x(t)-x(-t)\} + \frac{1}{2}\{x(t)+x(-t)\}
        \end{align*}
    So $\text{Ev\{x(t)\}}$ can be drawn as: \\
    \begin{figure}[h!]
    \centering
        \begin{tikzpicture}[scale=1.0]
           \begin{axis}[
          axis lines=middle,
          xlabel={$t$},
          ylabel={$\boldsymbol{y(t)}$},
          xtick={-4, -3, -2, ..., 4},
          ytick={-1, -0.5, 0, ..., 1},
          ymin=-1.5, ymax=1.5,
          xmin=-4, xmax=4,
          every axis x label/.style={at={(ticklabel* cs:1.05)}, anchor=northwest,},
          every axis y label/.style={at={(ticklabel* cs:1.05)}, anchor=south,},
          grid,
        ]
           \path[draw,line width=4pt] (-4,0) -- (-2,0)  -- (-1,0.5) -- (-1,0) -- (1,0) -- (1,-0.5) -- (2,0) -- (4,0);
           \end{axis}
        \end{tikzpicture}
        \caption{$t$ vs. $x(t)$.}
        \label{fig:q2}
    \end{figure}
    
    and $\text{Odd\{x(t)\}}$ can be drawn as: \\
    \begin{figure}[h!]
    \centering
        \begin{tikzpicture}[scale=1.0]
           \begin{axis}[
          axis lines=middle,
          xlabel={$t$},
          ylabel={$\boldsymbol{y(t)}$},
          xtick={-4, -3, -2, ..., 4},
          ytick={-1, -0.5, 0, ..., 1},
          ymin=-1.5, ymax=1.5,
          xmin=-4, xmax=4,
          every axis x label/.style={at={(ticklabel* cs:1.05)}, anchor=northwest,},
          every axis y label/.style={at={(ticklabel* cs:1.05)}, anchor=south,},
          grid,
        ]
           \path[draw,line width=4pt] (-4,0) -- (-2,0)  -- (-1,0.5) -- (-1,1) -- (1,1) -- (1,0.5) -- (2,0) -- (4,0);
           \end{axis}
        \end{tikzpicture}
        \caption{$t$ vs. $x(t)$.}
        \label{fig:q2}
    \end{figure}
    \end{enumerate}

\item %write the solution of q7
    \begin{enumerate}
    % Write your solutions in the following items.
    \item %write the solution of q7a
    $x(t) $ in terms of the unit step function is: \\
     $-3\mu(t-2)+5\mu(t-3)-3\mu(t-5)$ \\
    \item %write the solution of q7b
    Note that, $\frac{d(u(t))}{dt}=\delta(t)$
    So, $\frac{d}{dt} x(t) = -3\delta(t-2)+5\delta(t-3)-3\delta(t-5)$ 
    
    \begin{figure}[H]
            \centering
            \begin{tikzpicture}[scale=1.0] 
              \begin{axis}[
                  axis lines=middle,
                  xlabel={$t$},
                  ylabel={$\boldsymbol{\frac{d}{dt} x(t)}$},
                  xtick={ -1, 0, 1, ..., 5},
                  ytick={-3, -2, ..., 5},
                  ymin=-3, ymax=6,
                  xmin=-1, xmax=5,
                  every axis x label/.style={at={(ticklabel* cs:1.05)}, anchor=west,},
                  every axis y label/.style={at={(ticklabel* cs:1.05)}, anchor=south,},
                  grid,
                ]
                \addplot [ycomb, black, thick, mark=*] table [x={n}, y={xn}] {q7.dat};
              \end{axis}
            \end{tikzpicture}
            \caption{$t$ vs. $\frac{d}{dt} x(t)$.}
            \label{fig:q3}
        \end{figure}
            
    \end{enumerate}    

\item %write the solution of q8
    \begin{enumerate}
    % Write your solutions in the following items.
    \item %write the solution of q8a ########################################################
    \begin{enumerate}
        \item The system has memory. Proof:\\
        \(y[n]=x[3n-5]\)\\
        For \(n=1\ y[1]=x[-2]\).\\
    So it has to remember past values.Therefore system has memory.
        \item The system is stable since its amplitude is bounded and it does not varies due to input.
        \item The system depends on only the present outputs, so this system is causal.
        \item For a system to be linear, it needs to hold superposition property. Let $x_1$ and $x_2$ be two input signals:
            \begin{align*}
            y_1[n] & = x_1[3n-5] \\
            y_2[n] & = x_2[3n-5]
            \end{align*}
            When we add these up and multiply by some constants $a_1$ and $a_2$, we will have a $y_3$ as:
            \begin{align*}
            y_3[n]&= a_1 \times y_1[n] + a_2 \times y_2[n] \\
                  &= a_1 \times x_1[3n-5]+ a_2 \times x_2[3n-5]
            \end{align*}
            and on the other hand when we first perform addition and multiplication then put the signal as input to the system we will have a $y_3^{'}$ such as:
            \begin{align*}
                x_3[n]&=a_1\times x_1[3n-5] + a_2\times x_2[3n-5] \\
                y_3^{'}[n]&= x_3[n]\\
                &=a_1\times x_1[3n-5] + a_2\times x_2[3n-5]
            \end{align*}
            Since $y_3 = y_3'$ superposition property holds and system is linear.
        \item The system is invertible. Proof:
        \begin{align*}
        x[n] &= h^{-1}(y[n]) \\
        x[n] &= y[\frac{n+5}{3}]
        \end{align*}
        \item Checking the time in-variance:
        \begin{center}
        Let $x_1[3n-5] = x[3n-n_0-5]$ \\
         $y[n]= x_1[3n-5]$ \\
        So $y[n] = x[3n - n_0-5]$
        \end{center}
        On the other hand we have:
        \begin{align*}
        y^{'}[n] &= y[n-n_0] \\
        &= x[3n-3n_0-5]
        \end{align*}
        Since $y[n] \neq y^{'}[n]$ system is time variant. \\
        \end{enumerate}
    \item %write the solution of q8b########################################################
    \begin{enumerate}
        \item The system has memory. Proof:\\
        \(y(t)=x(3t-5)\)\\
        For \(n=1\ y(1)=x(-2)\).\\ So it has to remember past values.Therefore system has memory.
        \item The system is stable since its amplitude is bounded and it does not varies due to input.
        \item The system depends on the present or past outputs, so this system is causal.
        \item For a system to be linear, it needs to hold superposition property. Let $x_1$ and $x_2$ be two input signals:
           \begin{align*}
            y_1(t) & = x_1(3t-5) \\
            y_2(t) & = x_2(3t-5)
            \end{align*}
            When we add these up and multiply by some constants $a_1$ and $a_2$, we will have a $y_3$ as:
            \begin{align*}
            y_3(t)&= a_1 \times y_1(t) + a_2 \times y_2(t) \\
                  &= a_1 \times x_1(3t-5)+ a_2 \times x_2(3t-5)
            \end{align*}
            and on the other hand when we first perform addition and multiplication then put the signal as input to the system we will have a $y_3^{'}$ such as:
            \begin{align*}
                x_3(t)&=a_1\times x_1(3t-5) + a_2\times x_2(3t-5) \\
                y_3^{'}(t)&= x_3(t)\\
                &=a_1\times x_1(3t-5) + a_2\times x_2(3t-5)
            \end{align*}
            Since $y_3 = y_3'$ superposition property holds and system is linear.
        \item The system is invertible. Proof:
        \begin{align*}
        x(t) &= h^{-1}(y(n)) \\
        x(t) &= y(\frac{t+5}{3})
        \end{align*}
         \item Checking the time in-variance:
        \begin{center}
        Let $x_1(t) = x(3t-t_0-5)$ \\
         $y(t)= x_1(3t-5)$ \\
        So $y(t) = x(3t - t_0-5)$
        \end{center}
        On the other hand we have:
        \begin{align*}
        y^{'}(t) &= y(t-t_0) \\
        &= x(3t-3t_0-5)
        \end{align*}
        Since $y(t) \neq y^{'}(t)$ system is time variant. \\
        \end{enumerate}
    \item %write the solution of q8c########################################################
    \begin{enumerate}
        \item The system has memory. Proof:\\
        \(y(t)=tx(t-1)\)\\
        For \(n=1\ y(1)=x(0)\).\\ So it has to remember past values.Therefore system has memory.
        \item The system is not stable since its amplitude is not bounded and it varies due to input.
        \item The system depends on the present or past outputs, so this system is causal.
        \item For a system to be linear, it needs to hold superposition property. Let $x_1$ and $x_2$ be two input signals:
            \begin{align*}
            y_1(t) & =t \times x_1(t-1) \\
            y_2(t) & =t \times x_2(t-1)
            \end{align*}
            When we add these up and multiply by some constants $a_1$ and $a_2$, we will have a $y_3$ as:
            \begin{align*}
            y_3(t)&= a_1 \times y_1(t) + a_2 \times y_2(t) \\
                  &= a_1 \times t \times x_1(t-1)+ a_2 \times t \times x_2(t-1)
            \end{align*}
            and on the other hand when we first perform addition and multiplication then put the signal as input to the system we will have a $y_3^{'}$ such as:
            \begin{align*}
                x_3(t)&=a_1\times x_1(t-1) + a_2\times x_2(t-1) \\
                y_3^{'}(t)&=t \times x_3(t-1)\\
                &=a_1\times t \times x_1(t-1) + a_2\times t \times x_2(t-1)
            \end{align*}
            Since $y_3 = y_3'$ superposition property holds and system is linear.
        \item The system is not invertible. Proof:
        \begin{align*}
        x(t) &\neq h^{-1}(y(t)) \\
        x(t) &\neq \frac{y(t+1}{t}\\
        &For\ example\ for\ t=2;\\
        &x(2)=\frac{y(3)}{2}\ but\ y(3)=3\cdot x(2).
        \end{align*}Therefore the system is not invertible.
         \item Checking the time in-variance:
        \begin{center}
        Let $x_1(t-1) = x(t-t_0-1)$ \\
         $y(t)=t \times x_1(t-1)$ \\
        So $y(t) = t \times x(t - t_0-1)$
        \end{center}
        On the other hand we have:
        \begin{align*}
        y^{'}(t) &= y(t-t_0) \\
        &= (t - t_0) \times x(t-t_0-1)
        \end{align*}
        Since $y(t) \neq y^{'}(t)$ system is time variant. \\
        \end{enumerate} 
    \item %write the solution of q8d########################################################
    \begin{enumerate}
        \item The system has memory. Proof:\\
        \(y[n]=\)\(\sum_{k=1}^{\infty}x[n-k]\)\\
        For \(n=1\ y[1]=x[0]+x[-1]+\dots \)\\ So it has to remember past values.Therefore system has memory.
        \item The system is not stable since its amplitude is not bounded and it varies due to input.
        \item The system depends on the present or past outputs, so this system is causal.
        
        \end{enumerate}
    \end{enumerate}
    
\end{enumerate}
\end{document}

