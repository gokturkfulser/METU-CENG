\documentclass[10pt,a4paper, margin=1in]{article}
\usepackage{fullpage}
\usepackage{amsfonts, amsmath, pifont}
\usepackage{amsthm}
\usepackage{graphicx}
\usepackage{tkz-euclide}
\usepackage{tikz}
\usepackage{pgfplots}
\graphicspath{ {./images/} }
\usepackage{geometry}
 \geometry{
 a4paper,
 total={210mm,297mm},
 left=10mm,
 right=10mm,
 top=10mm,
 bottom=10mm,
 }
 % Write both of your names here. Fill exxxxxxx with your ceng mail address.
 \author{
  Karaman, Arda\\
  \texttt{e2237568@ceng.metu.edu.tr}
  \and
  Fulser, Göktürk\\
  \texttt{e2237386@ceng.metu.edu.tr}
}
\title{CENG 384 - Signals and Systems for Computer Engineers \\
Spring 2021 \\
Homework 3}

\begin{filecontents}{q1_a.dat}
 k   ak
 -1   0.5
 0   0.5
 1   0.5
 
\end{filecontents}
\begin{filecontents}{q1_b1.dat}
 k   ak
 -1   1
 0   1.5
 1   1
 
\end{filecontents}
\begin{filecontents}{q1_b2.dat}
 k   ak
 -1  1.5708
 0   0
 1   -1.5708
 
\end{filecontents}
\begin{filecontents}{q1_c1.dat}
 k   ak
 -2  0.5
 -1  1.118
 0   2
 1   1.118
 2   0.5
 
\end{filecontents}
\begin{filecontents}{q1_c2.dat}
 k   ak
 -2  -45
 -1  63.4
 0   0
 1   -63.4
 2   45
 
\end{filecontents}

\begin{filecontents}{q3_b.dat}
 k   ak
 0   1
 2   0.25
 4   0.5
 6   0.33
 
\end{filecontents}

\begin{filecontents}{q7_a1.dat}
  k  ak
 -4  2
 -3  1
 -2  0
 -1  1  
  0  2
  1  1
  2  0
  3  1
\end{filecontents}

\begin{filecontents}{q7_a2.dat}
  k  ak
 -4  0
 -3  2
 -2  0
 -1  2
  0  0
  1  2
  2  0
  3  2
\end{filecontents}

\begin{filecontents}{q7_b1.dat}
  k  ak
 -4  3
 -3  2
 -2  1
 -1  2
  0  3
  1  2
  2  1
  3  2
\end{filecontents}
 
\begin{filecontents}{q7_b2.dat}
  k  ak
 -4  0
 -3  2
 -2  0
 -1  1  
  0  0
  1  2
  2  0
  3  1
\end{filecontents}



\begin{document}
\maketitle



\noindent\rule{19cm}{1.2pt}

\begin{enumerate}

\item %write the solution of q1
    \begin{enumerate}
    % Write your solutions in the following items.
    \item %write the solution of q1a
     \begin{align*}
        x(t) = \frac{1}{2} + \cos{\omega_0t}
    \end{align*}
    The Euler formula provides us the representation of a periodic function in terms of harmonically related complex exponentials, as follows;
    \begin{align*}
        x(t) = \frac{1}{2} + \frac{1}{2}(e^{jw_0t} + e^{-jw_0t})
    \end{align*}
    Thus, the Fourier series coefficients of the function are,
    \begin{align*}
        a_0 = a_1 = a_{-1} = \frac{1}{2}, a_k = 0 \ for \ k\neq -1,0,1.
    \end{align*}
    The plot of the spectral coefficients,
    \begin{figure} [h!]
        \centering
        \begin{tikzpicture}[scale=1.0] 
          \begin{axis}[
              axis lines=middle,
              xlabel={$k$},
              ylabel={$\boldsymbol{a_k}$},
              xtick={ -2, -1,  ..., 2},
              ytick={ -0.5, 0, 0.5 , 1},
              ymin=-1/2, ymax=1,
              xmin=-2, xmax=2,
              every axis x label/.style={at={(ticklabel* cs:1.05)}, anchor=west,},
              every axis y label/.style={at={(ticklabel* cs:1.05)}, anchor=south,},
              grid,
            ]
            \addplot [ycomb, black, thick, mark=*] table [x={k}, y={ak}] {q1_a.dat};
          \end{axis}
        \end{tikzpicture}
        \caption{$k$ vs. $a_k$.}
        \label{fig:q2}
    \end{figure}
    \item %write the solution of q1b
    \begin{align*}
        y(t) = \frac{3}{2} + 2\cdot \sin{\omega_0t}
    \end{align*}
    The Euler formula provides us the representation of a periodic function in terms of harmonically related complex exponentials, as follows;
    \begin{align*}
        y(t) = \frac{3}{2} + \frac{(e^{jw_0t} - e^{-jw_0t})}{j}
    \end{align*}
    Thus, the Fourier series coefficients of the function are,
    \begin{align*}
        a_0 = \frac{3}{2},\ a_1 = \frac{1}{j} = -j,\ a_{-1} = \frac{-1}{j} = j,\ a_k = 0 \ for \ k\neq -1,0,1.
    \end{align*}
    So the magnitude, and phase spectrums of the coefficients:
    \begin{figure} [h!]
        \centering
        \begin{tikzpicture}[scale=1.0] 
          \begin{axis}[
              axis lines=middle,
              xlabel={$k$},
              ylabel={$\boldsymbol{|a_k|}$},
              xtick={ -2, -1,  ..., 2},
              ytick={  0, 0.5, 1, 1.5, 2},
              ymin=0, ymax=2,
              xmin=-2, xmax=2,
              every axis x label/.style={at={(ticklabel* cs:1.05)}, anchor=west,},
              every axis y label/.style={at={(ticklabel* cs:1.05)}, anchor=south,},
              grid,
            ]
            \addplot [ycomb, black, thick, mark=*] table [x={k}, y={ak}] {q1_b1.dat};
          \end{axis}
        \end{tikzpicture}
        \caption{Magnitude Spectrum}
        \label{fig:q2}
    \end{figure}
    
    \begin{figure} [h!]
        \centering
        \begin{tikzpicture}[scale=1.0] 
          \begin{axis}[
              axis lines=middle,
              xlabel={$k$},
              ylabel={$\boldsymbol{\theta_k}$},
              xtick={ -2, -1,  ..., 2},
              ytick={-1.5708, 0, 1.5708},
              yticklabels={
              $\frac{-\pi}{2}$, 0 , $\frac{\pi}{2}$
              },
              ymin=-1.5708, ymax=1.5708,
              xmin=-2, xmax=2,
              every axis x label/.style={at={(ticklabel* cs:1.05)}, anchor=west,},
              every axis y label/.style={at={(ticklabel* cs:1.05)}, anchor=south,},
              grid,
            ]
            \addplot [ycomb, black, thick, mark=*] table [x={k}, y={ak}] {q1_b2.dat};
          \end{axis}
        \end{tikzpicture}
        \caption{Phase Spectrum}
        \label{fig:q2}
    \end{figure}
    \item %write the solution of q1c
     \begin{align*}
        z(t) = x(t) + y(t) + \cos({2\omega_0 t+\frac{\pi}{4}})
    \end{align*}
    From part a, and part b :
    \begin{align*}
        x(t) = \frac{1}{2} + \frac{1}{2}(e^{jw_0t} + e^{-jw_0t})\\
        y(t) = \frac{3}{2} + \frac{(e^{jw_0t} - e^{-jw_0t})}{j}
    \end{align*}
    The Euler formula provides us the representation of a periodic function in terms of harmonically related complex exponentials, as follows;
    \begin{align*}
        \frac{1}{2}(e^{j(2\omega_0t+\frac{\pi}{4})}+e^{-j(2\omega_0t+\frac{\pi}{4})})
    \end{align*}
    \begin{align*}
        \frac{1}{2}(e^{j(2\omega_0t)\cdot}e^{\frac{j\pi}{4}}+e^{-j(2\omega_0t)\cdot}e^{\frac{-j\pi}{4}})
    \end{align*}
    From euler formula:
    \begin{align*}
        e^{\frac{j\pi}{4}} = \cos{\frac{\pi}{4}} + j\sin{\frac{\pi}{4}}
    \end{align*}
    \begin{align*}
        e^{\frac{j\pi}{4}} = \frac{\sqrt{2}}{2} + j\frac{\sqrt{2}}{2}
    \end{align*}
    So $z(t)$ equals;
    \begin{align*}
        z(t) = 2 + e^{j\omega_0t}(\frac{1}{2}-j) + e^{-j\omega_0t}(\frac{1}{2}+j) + e^{2j\omega_0t}(\frac{\sqrt{2}}{4}(1+j)) + e^{-2j\omega_0t}(\frac{\sqrt{2}}{4}(1-j))
    \end{align*}
    Thus, the Fourier series coefficients of the function are,
    \begin{align*}
        a_0 = 2,\ a_1 = \frac{1}{2}-j,\ a_{-1} = \frac{1}{2}+j,\ a_2 = \frac{\sqrt{2}}{4}(1+j),\ a_{-2} = \frac{\sqrt{2}}{4}(1-j) \
        a_k = 0 \ for \ k\neq -2,-1,0,1,2.
    \end{align*}
    So the magnitude, and phase spectrums of the coefficients:
    \begin{figure} [h!]
        \centering
        \begin{tikzpicture}[scale=1.0] 
          \begin{axis}[
              axis lines=middle,
              xlabel={$k$},
              ylabel={$\boldsymbol{|a_k|}$},
              xtick={-3, -2, -1,  ..., 3},
              ytick={0, 0.5, 1.118, 2},
              yticklabels={
              0, $\frac{1}{2}$ , $\frac{\sqrt{5}}{2}$ , 2
              },
              ymin=0, ymax=2,
              xmin=-3, xmax=3,
              every axis x label/.style={at={(ticklabel* cs:1.05)}, anchor=west,},
              every axis y label/.style={at={(ticklabel* cs:1.05)}, anchor=south,},
              grid,
            ]
            \addplot [ycomb, black, thick, mark=*] table [x={k}, y={ak}] {q1_c1.dat};
          \end{axis}
        \end{tikzpicture}
        \caption{Magnitude Spectrum}
        \label{fig:q2}
    \end{figure}
    \begin{figure} [h!]
        \centering
        \begin{tikzpicture}[scale=1.0] 
          \begin{axis}[
              axis lines=middle,
              xlabel={$k$},
              ylabel={$\boldsymbol{\theta(^\circ)}$},
              xtick={-3, -2, -1,  ..., 3},
              ytick={-63.4, -45, 0, 45, 63.4},
              ymin=-63.4, ymax=63.4,
              xmin=-3, xmax=3,
              every axis x label/.style={at={(ticklabel* cs:1.05)}, anchor=west,},
              every axis y label/.style={at={(ticklabel* cs:1.05)}, anchor=south,},
              grid,
            ]
            \addplot [ycomb, black, thick, mark=*] table [x={k}, y={ak}] {q1_c2.dat};
          \end{axis}
        \end{tikzpicture}
        \caption{Phase Spectrum}
        \label{fig:q2}
    \end{figure}
    \end{enumerate}

\item %write the solution of q2
\begin{align*}
    x(t) = \frac{A_0}{2} + \sum_{k=1}^{\infty} A_k\cos{k\omega t} + B_k\sin{k\omega t} \\
    A_0 = \frac{M\cdot T_1}{T}\\
    A_K = \frac{2}{T} \cdot \int^{T}_{0} x(t) cosk \omega_0 t dt \\
    A_K = \frac{2}{T} \cdot \int^{T_1}_{0} M cosk \omega_0 t dt += \int^{T}_{T_1} 0 cosk \omega_0 t dt\\
    A_K = \frac{2M}{T} \cdot [\frac{sink\omega_0 t}{k\omega_0}] + 0\\
    A_K = \frac{2M}{T} \cdot [\frac{sink 2\frac{\pi}{T}T_1}{k\frac{2\pi}{T}}] =   \\
    A_K = \frac{M}{k\pi}\cdot sin 2\pi k \frac{T_1}{T}\\
    A_K = \frac{M}{K \pi} sin(2\pi K \frac{T_1}{T})
\end{align*}

For $B_K$:
\begin{align*}
    B_K = \frac{2}{T} \cdot \int^{T}_{0} x(t) sink \omega_0 t dt \\
    = \frac{2}{T} \cdot \int^{T_1}_{0} M sink \omega_0 t dt \\
    = \frac{2M}{T} \cdot  [\frac{- cos k \omega_0 t }{k \omega_0 t }] \\
    = \frac{2M}{T} \cdot [\frac{- cos k \frac{2\pi}{T} T_1 }{K \frac{2\pi}{T} } + \frac{1}{K \frac{2\pi}{T}}] \\
    B_K = \frac{M}{K\pi} \cdot [1 - cos(2\pi K \frac{ T_1}{T})]
\end{align*}


\item %write the solution of q3  
    \begin{enumerate}
    % Write your solutions in the following items.
    \item %write the solution of q3a
    \includegraphics[scale=0.35]{q3a.jpeg}
    \item %write the solution of q3b
    \begin{align*}
        x(t) = 1 + \frac{1}{2}\cos{2\pi t} + \cos{4\pi t} + \frac{2}{3} \cos{6\pi t}
    \end{align*}
    The Euler formula provides us the representation of a periodic function in terms of harmonically related complex exponentials, as follows;
    \begin{align*}
        x(t) = 1 + \frac{1}{4}(e^{j2\pi t} + e^{-j2\pi t})+\frac{1}{2}(e^{j4\pi t} + e^{-j4\pi t})+\frac{1}{3}(e^{j6\pi t} + e^{-j6\pi t})
    \end{align*}
    Thus, the Fourier series coefficients of the function are,
    \begin{align*}
        a_0 = 1,\ a_2 = \frac{1}{4},\ a_4 = \frac{1}{2},\ a_6 = \frac{1}{3},\ a_k = 0 \ for \ k\neq 0,2,4,6.
    \end{align*}
    The plot of the spectral coefficients,
    \begin{figure} [h!]
        \centering
        \begin{tikzpicture}[scale=1.0] 
          \begin{axis}[
              axis lines=middle,
              xlabel={$k$},
              ylabel={$\boldsymbol{a_k}$},
              xtick={ 0, 1,  ..., 7},
              ytick={ 0, 0.25 , 1/3 , 0.50, 1},
              yticklabels={0,
              $\frac{1}{4}$, $\frac{1}{3}$ , $\frac{1}{2}$ , 1
              },
              ymin=0, ymax=1,
              xmin=0, xmax=7,
              every axis x label/.style={at={(ticklabel* cs:1.05)}, anchor=west,},
              every axis y label/.style={at={(ticklabel* cs:1.05)}, anchor=south,},
              grid,
            ]
            \addplot [ycomb, black, thick, mark=*] table [x={k}, y={ak}] {q3_b.dat};
          \end{axis}
        \end{tikzpicture}
        \caption{$k$ vs. $a_k$.}
        \label{fig:q2}
    \end{figure}
    \item %write the solution of q3c
    Recall that, each spectral coefficient, $a_k$ shows the amount of the corresponding harmonic frequency in the signal, $x(t)$. For small periods, the signal, $x(t)$ has relatively less low frequency components, compared to the other signals. As we increase the period of the signal, the low frequency components increase and the rate of change of the spectral coefficients decrease. Investigation of the behaviour of the spectral coefficients show the amount of each frequency component relative to each other, which makes the signal. This is why we call the plot, $a_k vs. k$ as spectrum, meaning the band of frequencies, in $x(t)$.
    \item %write the solution of q3d
    \end{enumerate}

\item %write the solution of q4
    \begin{enumerate}
    % Write your solutions in the following items.
    \item %write the solution of q4a
    \item %write the solution of q4b
    \end{enumerate}

\item %write the solution of q5
    \begin{enumerate}
    % Write your solutions in the following items.
    \item %write the solution of q5a
    The Discrete Fourier Series pair is:
    \begin{align*}
        x[n] = \sum\limits_{k=<n>} a_{k} \cdot e^{j k\frac{2\pi}{N}} \\
        a_k = \frac{1}{N} \sum\limits_{k=<n>} x[n] \cdot e^{- j k\frac{2\pi}{N}}
    \end{align*}
    
    Given $ x[n] = sin(\frac{\pi}{2} \cdot n)  $ , here $N = 4$.
    \begin{align*}
        x[n] = \frac{ e^{j\frac{\pi}{4}n} - e^{j\frac{\pi}{4}n} }{ 2j } \\
        a_1 = \frac{1}{2j} e^{j} \\
        a_{-1} = - \frac{1}{2j} \cdot e^{-j} \quad \text{and the rest is zero.}
    \end{align*} 
    \item %write the solution of q5b
        The Discrete Fourier Series pair is:
    \begin{align*}
        x[n] = \sum\limits_{k=<n>} a_{k} \cdot e^{j k\frac{2\pi}{N}} \\
        a_k = \frac{1}{N} \sum\limits_{k=<n>} x[n] \cdot e^{- j k\frac{2\pi}{N}}
    \end{align*}
    
    Given $ y[n] = 1 + cos(\frac{\pi}{2} \cdot n)  $ , here $N = 4$.
    \begin{align*}
        y[n] = 1 + \frac{ e^{j\frac{\pi}{4}n} + e^{-j\frac{\pi}{4}n} }{ 2 } \\
        b_0 = 1 \\
        b_1 = \frac{1}{2} \\
        b_{-1} = \frac{1}{2} \quad \text{and the rest is zero.}
    \end{align*}
    \item %write the solution of q5c
    \begin{align*}
        z[n] = x[n] \cdot y[n]
    \end{align*}
    From multiplication property, the Fourier coefficients of $z[n]$ convolution of Fourier series coefficients of $x[n]$ and $y[n]$. \\
    \begin{align*}
        c_0 = a_0 \cdot b_0 + a_1 \cdot b_{-1} +  a_2 \cdot b_{-2} + a_3 \cdot b_{-3} + a_4 \cdot b_{-4} + a_5 \cdot b_{-5} \\
        c_0 = \frac{1}{4j}
    \end{align*}
    We have non-zero values at $a_0, a_1, b_0, b_{-1}$.
    \item %write the solution of q5d
    \begin{align*}
        z[n] = x[n] \cdot y[n] \\
        = sin(\frac{\pi}{2} n) \cdot [1 + cos(\frac{\pi}{2} n)] \\
        = sin(\frac{\pi}{2} n) + sin(\frac{\pi}{2} n) \cdot cos(\frac{\pi}{2} n) \\
        ** sin(\alpha) \cdot cos(\beta) = \frac{1}{2} \cdot [sin(\alpha + \beta) + sin(\alpha - \beta)] \\
        z[n] = sin(\frac{\pi}{2} n) + \frac{1}{2} \cdot [sin(\pi n) + sin(0)] \\
        z[n] = sin(\frac{\pi}{2} n) \\
        z[n] =  \frac{ e^{j\frac{\pi}{2}n} - e^{j\frac{\pi}{2}n} }{ 2j }
    \end{align*}
    From the part a) :
    \begin{align*}
        c_1 = \frac{1}{2j} e^{j} \\
        c_{-1} = - \frac{1}{2j} \cdot e^{-j}
    \end{align*}
    \end{enumerate}

\item %write the solution of q6
\begin{align*}
     \omega_0 = \frac{\pi}{6}, \ N = \frac{2\pi}{\omega_0}
\end{align*}
\begin{align*}
    N = 12
\end{align*}
\begin{align*}
    a_k = \frac{1}{12} \sum_{n=0}^{11} x[n]e^{-jk(\frac{\pi}{6})n}
\end{align*}
\begin{align*}
    a_k = \cos{\frac{k\pi}{6}} + \sin{\frac{5k\pi}{6}}
\end{align*}
\begin{align*}
    a_k = \frac{1}{2}e^{j\frac{k\pi}{6}} + \frac{1}{2}e^{-j\frac{k\pi}{6}} + \frac{1}{2j}e^{j\frac{5k\pi}{6}} - \frac{1}{2j}e^{-j\frac{5k\pi}{6}}
\end{align*}
Hence by comparing equations we can write x[n].
\begin{align*}
    x[n] = 6\delta[n-1] + 6\delta[n-11] -6j\delta[n-5] +6j\delta[n-7], \ 0\leq n \leq 11
\end{align*}
\item %write the solution of q7
    \begin{enumerate}
    % Write your solutions in the following items.
    \item 
    
    	To find Fourier series coefficients of $x[n]$ we will use following formula for 1 period:
		\begin{center}
		$a_k = \frac{1}{N}\sum_{n=<N>} x[n]e^{-jkw_0n}$ \\
		\end{center}
		Now we need $N$ and $w_0=\frac{2\pi}{N}$. From the graph of $x[n]$ we can deduce that $N=4$ and $w_0=\frac{\pi}{2}$. To find the Fourier Series coefficients:
		\begin{align*}
		a_k &= \frac{1}{4}\sum_{n=0}^{3} x[n]e^{-jk\frac{\pi}{2}n} \\
			&= \frac{1}{4}(x[0] + x[1]e^{-jk\frac{\pi}{2}} + x[2]e^{-2jk\frac{\pi}{2}} + x[3]e^{-3jk\frac{\pi}{2}}) \\
			&= 0 + \frac{1}{4}(cos(k\frac{\pi}{2}) - jsin(k\frac{\pi}{2})) + \frac{1}{2}(cos(k\pi) - jsin(k\pi)) + \frac{1}{4}(cos(3k\frac{\pi}{2}) - jsin(3k\frac{\pi}{2}))
		\end{align*}
		From above:
		\begin{align*}
		a_0 &= \frac{1}{4} + \frac{1}{2} + \frac{1}{4} = 1 \\
		a_1 &= -\frac{j}{4} - \frac{1}{2} + \frac{j}{4} = -\frac{1}{2} \\
		a_2 &= -\frac{1}{4} + \frac{1}{2} -\frac{1}{4} = 0 \\
		a_3 &= \frac{j}{4} - \frac{1}{2} -\frac{j}{4} = -\frac{1}{2}			
		\end{align*}
		In conclusion we can say: $a_n = a_{n+4} = a_{n-4}$, the other coefficients can be found using this. \\
		We can plot the magnitude spectrum of the coefficients: \\
    
    \begin{figure}[h!]
    \centering
    \ldots
    \begin{tikzpicture}[scale=1.0] 
      \begin{axis}[
          axis lines=middle,
          xlabel={$k$},
          ylabel={$|a_k|$},
          xtick={ -4, -3, ..., 3},
          ytick={0,1,2,3},
          yticklabels={0 , $\frac{1}{2}$, 1, $\frac{3}{2}$},
          ymin=0, ymax=3,
          xmin=-4, xmax=3,
          every axis x label/.style={at={(ticklabel* cs:1.05)}, anchor=west,},
          every axis y label/.style={at={(ticklabel* cs:1.05)}, anchor=south,},
          grid,
        ]
        \addplot [ycomb, black, thick, mark=*] table [x={k}, y={ak}] {q7_a1.dat};
      \end{axis}
    \end{tikzpicture}
    \ldots
    \caption{$k$ vs. $|a_k|$.}
    \label{fig:q7_a1}
\end{figure}

		And the phase spectrum of the coefficients: 
\begin{figure}[h!]
    \centering
    \ldots
    \begin{tikzpicture}[scale=1.0] 
      \begin{axis}[
          axis lines=middle,
          xlabel={$k$},
          ylabel={$\angle a_k$},
          xtick={ -4, -3, ..., 3},
          ytick={-3, -2, ..., 3},
          yticklabels={$-\frac{3\pi}{2}$, $-\pi$, $-\frac{\pi}{2}$, 0, $\frac{\pi}{2}$, $\pi$, $\frac{3\pi}{2}$},
          ymin=-3, ymax=3,
          xmin=-4, xmax=3,
          every axis x label/.style={at={(ticklabel* cs:1.05)}, anchor=west,},
          every axis y label/.style={at={(ticklabel* cs:1.05)}, anchor=south,},
          grid,
        ]
        \addplot [ycomb, black, thick, mark=*] table [x={k}, y={ak}] {q7_a2.dat};
      \end{axis}
    \end{tikzpicture}
    \ldots
    \caption{$k$ vs. $\angle a_k$.}
    \label{fig:q7_a2}
\end{figure}

    \item %write the solution of q7b
    \begin{enumerate}
			\item
				We need to add a negative impulse (i.e $-\delta(n)$) at every $n+1$ th point to ensure periodicity of $y[n]$. So we would end up something such as:
				\begin{center}
				$y[n] = x[n] - \sum_{k=-\infty}^{\infty} \delta(n+1 - 4k)$ \\
				\end{center}
			\item
				To find Fourier series coefficients of $y[n]$:
			\begin{center}
			$a_k = \frac{1}{N}\sum_{n=<N>} x[n]e^{-jkw_0n}$ \\
			\end{center}
			
			And to be able to find them using this formula we need $N$ and $w_0=\frac{2\pi}{N}$. From the graph of $y[n]$ we can see that $N=4$ and $w_0=\frac{\pi}{2}$. To find the Fourier Series coefficients:
			\begin{align*}
			a_k &= \frac{1}{4}\sum_{n=0}^{3} x[n]e^{-jk\frac{\pi}{2}n} \\
				&= \frac{1}{4}(x[0] + x[1]e^{-jk\frac{\pi}{2}} + x[2]e^{-2jk\frac{\pi}{2}} + x[3]) \\
				&= 0 + \frac{1}{4}(cos(k\frac{\pi}{2}) - jsin(k\frac{\pi}{2})) + \frac{1}{2}(cos(k\pi) - jsin(k\pi)) + 0
			\end{align*}
			From above:
			\begin{align*}
			a_0 &= \frac{1}{4} + \frac{1}{2} = \frac{3}{4} \\
			a_1 &= -\frac{j}{4} - \frac{1}{2} = -\frac{1}{4}(j+2) \\
			a_2 &= -\frac{1}{4} + \frac{1}{2} = \frac{1}{4} \\
			a_3 &= \frac{j}{4} - \frac{1}{2} = \frac{1}{4}(j-2) 		
			\end{align*}
			From periodicity we can say: $a_n = a_{n+4} = a_{n-4}$, so other coefficients can be found from this. \\\\
			The magnitude spectrum of the coefficients: 
\begin{figure}[H!]
    \centering
    \ldots
    \begin{tikzpicture}[scale=1.0] 
      \begin{axis}[
          axis lines=middle,
          xlabel={$k$},
          ylabel={$|a_k|$},
          xtick={ -4, -3, ..., 3},
          ytick={0,1,2,3},
          yticklabels={0, $\frac{1}{4}$, $\sqrt{\frac{5}{16}}$, $\frac{3}{4}$},
          ymin=0, ymax=3,
          xmin=-4, xmax=3,
          every axis x label/.style={at={(ticklabel* cs:1.05)}, anchor=west,},
          every axis y label/.style={at={(ticklabel* cs:1.05)}, anchor=south,},
          grid,
        ]
        \addplot [ycomb, black, thick, mark=*] table [x={k}, y={ak}] {q7_b1.dat};
      \end{axis}
    \end{tikzpicture}
    \ldots
    \caption{$k$ vs. $|a_k|$.}
    \label{fig:q1_b1}
\end{figure}
			The phase spectrum of the coefficients: 
\begin{figure}[H!]
    \centering
    \ldots
    \begin{tikzpicture}[scale=1.0] 
      \begin{axis}[
          axis lines=middle,
          xlabel={$k$},
          ylabel={$\angle a_k$},
          xtick={ -4, -3, ..., 3},
          ytick={-2, -1, ..., 2},
		  yticklabels={$-1.14\pi$, $-0.85\pi$, 0, $0.85\pi$, $1.14\pi$},          
          ymin=-2, ymax=2,
          xmin=-4, xmax=3,
          every axis x label/.style={at={(ticklabel* cs:1.05)}, anchor=west,},
          every axis y label/.style={at={(ticklabel* cs:1.05)}, anchor=south,},
          grid,
        ]
        \addplot [ycomb, black, thick, mark=*] table [x={k}, y={ak}] {q7_b2.dat};
      \end{axis}
    \end{tikzpicture}
    \ldots
    \caption{$k$ vs. $\angle a_k$.}
    \label{fig:q1_b2}
\end{figure}
		\end{enumerate}
	\end{enumerate}	   
    

\end{enumerate}
\end{document}

